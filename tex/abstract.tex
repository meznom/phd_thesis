\chapter*{Abstract}

\begin{doublespace}
We undertake an in-depth numerical study of quantum-dot cellular automata (QCA),
a beyond-CMOS computing paradigm which represents bits as bistable charge
distributions in cells consisting of quantum dots. Using semi-realistic but
material-independent modelling, we characterize the building blocks of QCA
circuits in as detailed and unbiased a manner as possible.  Starting from an
extended Hubbard model, and introducing two controlled Hilbert space truncations
whose limits we study and understand, we use exact diagonalization to calculate
time-independent properties of small systems. We derive a transverse-field Ising
model as an effective description for QCA devices, but find that it is only
valid in a restrictive parameter range. We demonstrate that the commonly used
intercellular Hartree approximation is inadequate and gives results that are
qualitatively incorrect. In contrast to previous work, we show that the
response between pairs of adjacent cells is linear and does not exhibit gain.
Non-linearity and gain only emerge in response to static-charge input cells that
have no quantum dynamics of their own. As a consequence, QCA circuits cannot
retain a logic state in the thermodynamic limit, and there is a maximum circuit
size set by the system's parameters. Overall, QCA as a computing architecture is
seen to be more fragile than previously thought. We establish charge neutral
cells as a strict requirement for QCA operation. We identify parameter bounds
for functional devices: small cell-cell distances, moderate temperatures, and
large Coulomb energy scales are necessary.
\end{doublespace}
