\chapter{Conclusion}

We have undertaken the first in-depth numerical study of quantum-dot cellular
automata (QCA), a beyond-CMOS computing paradigm which represents binary state
as bistable charge distributions in cells consisting of several quantum dots. We
have concentrated on the time-independent properties of small and simple
structures, such as horizontal lines of cells, but strove to characterize them
as detailed and unbiased as possible, in a material-agnostic but semi-realistic
manner. Starting from an extended Hubbard model, our exact diagonalization
calculations avoid commonly used but problematic approximations, such as the
inter-cellular Hartree approximation, and instead introduce two controlled
Hilbert space truncations: the fixed charge and the bond model. We studied the
limits of these truncations and established the parameter range where they are
valid. We were the first to cleanly derive an effective transverse-field Ising
model for the QCA approach---another commonly used approximation---and
explicitly show in which limits it is valid, which we found to be quite
restrictive.

In contrast to previously published results, we found that the cell-cell response
function is linear and does not exhibit gain. In our calculations, gain and
non-linearity, hitherto claimed as important QCA characteristics, are only
observed in the response to static-charge input driver cells. As a consequence,
QCA systems cannot retain finite polarization---a definite logic state---in the
thermodynamic limit. In practical terms, the maximum size of a QCA device is
finite and determined by the quality of the cell-cell response. The cell-cell
response characteristic is universal for all system parameters, including zero
and finite temperatures. The absence of gain has profound implications for logic
applications---QCA's raison d'\^etre---which, at least for traditional CMOS-style
computing architectures, require binary switches with gain. Implicitly, our
findings indicate that the inter-cellular Hartree approximation is incorrect
and its results cannot be trusted.

We have identified charge neutral cells as a strict requirement for operational
QCA devices. Additionally, we have established lower limits for a functional QCA
system's parameters. Generally, short cell-cell distances are desirable for
optimal operation. The lower physical limit for the distance is cells placed one
cell apart, where inter-cell hopping starts to compete with intra-cell hopping,
and the upper limit depends on the nearest-neighbour Coulomb energy $V_1$ and
the temperature, where larger Coulomb terms and lower temperature increase the
operational distance range. For the Coulombic energy scale we identified an
absolute lower bound of $V_1/t > 20$ (in units of the hopping $t$), a limit that
is significantly increased at temperatures that are not very small. These
requirements---charge neutrality and large $V_1/t$ ratios---pose severe
challenges for experimental realizations of the QCA approach. The quality
of the cell-cell response increases with increasing Coulomb scale $V_1$. It is
in this limit, where additionally temperatures are not too high and cell-cell
distances not too large, that practically sufficiently large system sizes are
attainable for QCA. For a chosen set of system parameters, we showed explicitly
that devices with tens of thousands of cells are feasible.

It has to be noted that our characterization and results only apply to QCA
implementations that can be described, with good accuracy, by the extended
Hubbard model. Molecular QCA realizations, for example, are expected to behave
differently. Similarly, the aluminum island system, which has seen the most
experimental work on QCA so far, uses quantum dots with micrometer diameters and
is therefore not necessarily well-represented by a Hubbard model. In contrast,
our findings should be applicable to atomic silicon quantum dots and other truly
molecular-scale semiconductor-based experiments.

% outlook

% majority gate (?)
% only for systems covered by the extended Hubbard model
