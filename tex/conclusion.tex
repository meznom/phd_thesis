\chapter{Conclusion}

\glsreset{QCA}
We have undertaken the first in-depth numerical study of \cgls{QCA}, a
beyond-\cgls{CMOS} computing paradigm which represents binary states as bistable
charge distributions in cells consisting of several quantum dots. We have
concentrated on the time-independent properties of small and simple structures,
such as horizontal lines of cells, but have striven to characterize them with as
much detail and as little bias as possible, in a material-independent but
semi-realistic manner. Starting from an extended Hubbard model, our exact
diagonalization calculations avoid commonly used but problematic approximations,
such as the intercellular Hartree approximation, and instead introduce two
controlled Hilbert space truncations: the fixed-charge and the bond model. We
studied the limits of these truncations and established the parameter ranges
where they are valid. We were the first to derive (rather than presume) an
effective transverse-field Ising model for the \cgls{QCA} approach---another
commonly used approximation---and to establish the parameter range in which it
can be used, which we observed to be quite restrictive.

In contradiction to previously published results, we found that the cell-cell
response function is linear and does not exhibit gain \cite{ritter2014}. In our
calculations, gain and non-linearity, hitherto claimed as important \cgls{QCA}
characteristics, are only observed in the response to static-charge input driver
cells. As a consequence, \cgls{QCA} systems cannot retain finite
polarization---a definite logic state---in the thermodynamic limit. In practical
terms, this limits the size of \cgls{QCA} devices where the maximum size is
determined by the quality of the cell-cell response. The observed cell-cell
response characteristic is universal for all system parameters, including zero
and finite temperatures. The absence of gain makes \cgls{QCA} a less robust
scheme overall, and has profound implications for logic
applications---\cgls{QCA}'s raison d'\^etre---which, at least for traditional
\cgls{CMOS}-style computing architectures, require binary switches with gain.
Implicitly, our findings indicate that the intercellular Hartree approximation
is incorrect and its results cannot be trusted.

We have identified charge neutral cells as a strict requirement for operational
\cgls{QCA} devices. Additionally, we have established parameter bounds for a
functional system. Generally, short cell-cell distances are desirable for
optimal operation, but cells cannot be placed closer together than one cell
apart, for otherwise cells are no longer distinct physical entities. The upper
limit of the operational range of cell-cell distances depends on the
nearest-neighbour Coulomb energy $V_1$ and the temperature, with larger Coulomb
terms and lower temperatures increasing the range. For the Coulombic energy
scale, we identified an absolute lower bound of $V_1/t > 20$ (in units of the
hopping $t$), a value that increases significantly at moderate and high
temperatures. The cell-cell response improves with increasing Coulomb scale
$V_1$. It is in this limit, where additionally temperatures are not too high and
cell-cell distances not too large, that systems of a size practical for building
extended circuits are attainable for \cgls{QCA}. For a chosen set of system
parameters, we showed explicitly that devices with tens of thousands of cells
are feasible. Nonetheless, the identified requirements---charge neutrality and
large $V_1/t$ ratios---pose potentially severe challenges for experimental
realizations of the \cgls{QCA} approach.

It has to be noted that our characterization and results only apply to
\cgls{QCA} implementations that can be described, with good accuracy, by the
extended Hubbard model. Molecular \cgls{QCA} realizations, for example, are
expected to behave differently. Similarly, the aluminum island system, which has
seen the most experimental work on \cgls{QCA} so far, uses quantum dots with
micrometer diameters and is therefore not necessarily well-represented by a
Hubbard model. In contrast, our findings should be applicable to atomic silicon
quantum dots and other truly molecular-scale semiconductor-based devices.

There is room for more numerical work on the \cgls{QCA} approach. The
semi-realistic modelling and simulation of system dynamics is the most important
outstanding aspect for an exhaustive evaluation of \cgls{QCA} as a
beyond-\cgls{CMOS} technology. For small systems, our exact diagonalization
method can be extended to the calculation of time-resolved properties. Notably,
this requires the inclusion of a sufficiently accurate dissipative term. The
calculation of the switching time of the majority gate and the signal
transmission time of wires will then allow one to estimate the overall
operational time-scale of \cgls{QCA} devices and therefore permit more in-depth
comparisons to \cgls{CMOS} technology and other proposed novel computing
architectures.

Within the Ising approximation, stochastic series expansion Monte Carlo
techniques can be used for numerical simulations of the \cgls{QCA} approach and
would make much larger system sizes computationally accessible. While the Ising
approximation is only valid in restricted and rather extreme parameter regimes,
this would provide an avenue to study large-scale phenomena such as the design
of complex \cgls{QCA} circuits. So far, this has been the domain of
intercellular Hartree calculations, which we have proven to be deeply flawed.

A third possibility for future numerical work are models that are closer to
specific material systems. For example, for the atomic silicon quantum dots
\emph{ab initio} calculations shed light on this material's properties on a very
small scale. We have now provided a \cgls{QCA} characterization that is general
and therefore relatively abstract. Informed by \emph{ab initio} estimates, our
Hubbard model could be extended to include more material-specific
characteristics, for example a screening term or a detailed, non-isotropic
hopping term. As such modelling becomes very difficult very quickly, it should
best be pursued in tandem with experiments that allow one to benchmark and
verify the theoretical predictions.

In our opinion, the most promising path for \cgls{QCA} in the near future lies
in the experimental domain. We have identified lower bounds for \cgls{QCA}
systems' parameters that pose a challenge for experimental realizations,
particularly the large $V_1/t$ ratios. As a next step, experiments could test
whether those parameter requirements are achievable and whether small \cgls{QCA}
devices can be made to work at a basic, time-independent level. In particular,
the atomic silicon quantum dot fabrication capabilities have improved to the
point where the reliable manufacturing of small to medium sized structures is
within reach. We propose that for a line of cells, static signal transmission
can be tested by setting an input with an external static charge and sensing the
resulting charge distribution in the cells. Presently, for these systems the
strongly perturbative scanning tunnelling microscope measurement process is a
challenge, but this will surely be resolved with further work. The observation
of a signal transmission charge pattern in a line of ten cells, for example,
would establish that \cgls{QCA} does work, in principle, for the atomic silicon
quantum dot system and mark a major breakthrough for the \cgls{QCA} approach.
