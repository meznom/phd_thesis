\chapter{Introduction}

The rise of electronic information technology has been one of the main drivers
of economical and societal change over the past seventy years. Computers have
flown us to the moon, trade stocks, diagnose illnesses, and even run simulations
of quantum spin systems. The advent of the internet, invented some 25 years ago,
and the relentless march of an army of mobile gadgets, from the venerable
notebook, over the smart phone and tablet, to wearable tech of all forms and
colours, together with readily available mobile data connections, is changing
the way we communicate, socialize, read, write and think. The benefits of
information technology are so multifaceted and ubiquitous that it is easy to
take them for granted. Yet as we ask for faster, more functionally rich, and
lighter devices, the data centres that feed us our cloud streams have developed
a great hunger for energy. And if the internet of things is supposed to happen,
it surely needs more energy-efficient devices than the phones that we always
keep in sight of a power outlet. The desire to build functional and at the same
time more power-efficient computing technology has led to efforts at all levels
of the technology stack, from the data centres, to the processor architectures,
to better and more parallel algorithms. Digital circuitry and specifically the
transistor, which underpins all of modern day information technology, have not
been an exception.

%%%%%%%%%%%%%%%%%%%%%%%%%%%%%%%%%%%%%%%%%%%%%%%%%%%%%%%%%%%%%%%%%%%%%%%%%%%%%%%%

The penetration of computing technology into all aspects of modern life has been
fuelled by the incredible success of the complementary metal-oxide semiconductor
(CMOS) integrated circuit. Computer chips have become ever more cheaper,
smaller, power-efficient, and at the same time much more capable. But there is a
growing concern that CMOS is close to its scaling limits---that it can no longer
become ever faster and cheaper. CMOS uses two complementary n-type and p-type
metal-oxide-semiconductor field-effect transistors (MOSFETs), greatly reducing
power consumption compared to older technologies, such as n-type
metal-oxide-semiconductor (NMOS) logic. Invented in the early 1960s, the CMOS
integrated circuit has since seen the number of transistors per chip double
roughly every two years, an exponential growth predicted by Gordon Moore and
hence known as Moore's law \cite{moore1965cramming}. Whereas in the eighties
feature sizes were on the order of micrometres, today's processors use a 22 nm
process and integrate billions of transistors on a single chip
\cite{bohr2011evolution}. One of the main reasons for the relentless
miniaturization of CMOS technology is cost reduction. In the manufacturing
process, the cost is dominantly set per wafer---the slab of pure crystalline
silicon used as the device substrate.  Therefore, chips become cheaper by either
increasing the size of the wafer---current silicon wafers are typically 30 cm in
diameter---or by increasing the device density and thus by smaller feature
sizes. Obviously, higher device density means more functionality per same-area
chip. But smaller feature sizes also, in principle, allow for shorter switching
times and reduced switching energy, and thus faster and more power-efficient
devices. In the past, obstacles that seemed to inhibit the continued downscaling
were time and again overcome by scientific and engineering ingenuity, and the
exponential growth predicted by Moore's law has been kept pace with.
Technological innovation has been fuelled and financed by consumer demand for
more capable and functionally rich devices.

Historically, feature size scaling was limited by the photolithographic process
used to manufacture semiconductor integrated circuits. But advances in
fabrication technology, such as 193 nm immersion lithography and double
patterning, have pushed below the 32 nm mark, with 10 nm deemed possible.
Beyond, extreme ultraviolet lithography is being developed, promising even
smaller feature sizes. These dimensions approach the atomic scale and further
downsizing is increasingly inhibited by the fundamental physical limits of the
MOSFET. Simply put, the smaller the transistor in size, the higher the leakage
currents. There are several leakage channels. If the distance between source and
drain becomes too small, then electrons can tunnel through the channel region
regardless of the gate barrier, and it has been estimated that 5.9 nm is the
minimal gate dimension before this tunnelling current becomes substantial
\cite{cavin2012science}. Similarly, the gate voltage has to shrink with
shrinking feature sizes, which increases the subthreshold current through the
channel region. Lastly, if the gate oxide layer becomes too thin, electrons can
tunnel from the gate to the drain, again leading to a leakage current. Some of
these problems can be mitigated by technological advances. For example, current
generation microprocessors substitute the traditionally used silicon dioxide
with materials with higher relative permittivities, such as hafnium oxide, for
the gate oxide, allowing thicker oxide layers and thus decreasing electron
tunnelling. But overall, smaller feature sizes, which lead to faster switching
times and higher device densities, significantly increase leakage currents.
Already, leakage is a substantial part of the total power consumption of current
devices.

The International Technology Roadmap for Seminconductors (ITRS) \cite{itrs2011}
maps out the pace of future CMOS miniaturization and estimates that feature size
and voltage scaling can continue for one or two decades, before reaching its
absolute lower limit. Even with the scaling limits approaching, CMOS is still a
very viable technology with several strategies for future improvements. On the
MOSFET level, specialized FETs for specific applications could be used, possibly
on the same chip. For example, if speed is paramount then transistors with short
switching times but high leakage, and therefore high power consumption, can be
employed. Conversely, for power-conscious applications, slower transistors with
larger feature sizes but less leakage would be preferable
\cite{cavin2012science}. On a higher level, architecture and circuit design
could---and this is already done to some extent---work around the changed
electronic characteristics of downscaled devices. For manufacturing, higher
parallelism in fabrication, e.g.~larger wafers, could cut down costs. More
clever packaging, for example by stacking circuits on top of each other, could
increase device density further. Lastly, the integration of different and
complementary technologies with CMOS directly on the chip holds significant
promise for future applications. Combining CMOS circuitry with optical devices,
such as waveguides, detectors, LEDs, and Lasers, or radio frequency, or
micro-electro-mechanical systems (MEMs), to name just a few possibilities, would
all yield devices with richer functionality. Eventually, however, merely pushing
CMOS further will not be enough, and, consequently, considerable effort has been
put into the search and development of completely new computing paradigms that
could one day replace CMOS technology. Even if an emerging new computing
technology could not compete with CMOS in all aspects, it could, conceivably, be
used for specific applications, e.g.~memory, and thus complement traditional
circuitry.

%%%%%%%%%%%%%%%%%%%%%%%%%%%%%%%%%%%%%%%%%%%%%%%%%%%%%%%%%%%%%%%%%%%%%%%%%%%%%%

There is no shortage of ideas for novel computing principles and architectures
to replace or complement the existing technology \cite{cavin2012science}
\cite{bernstein2010device}. Broadly speaking, these ideas fall into three
categories. First, some device proposals seek to incrementally improve the
MOSFET. They might exploit better materials, or be based on other physical
principles internally, but show the same characteristics and outside
functionality as the transistor. They would be a drop-in replacement for
MOSFETs, and the computing architecture would remain otherwise unchanged.
Second, devices have been suggested that implement Boolean logic but use
different physical properties to store and communicate the binary state and
might, as a consequence, allow different architectural designs that better and
more efficiently exploit their specific characteristic properties. Lastly, some
ideas explore the radical departure from the existing computing architecture.
They do not necessarily strive to realize Boolean logic and include examples
such as quantum computing and neuromorphic computing, that is, computing based
on neural networks or otherwise inspired by nature \cite{mead1990neuromorphic}
\cite{schemmel2010wafer} \cite{furber2012overview}. These proposals are at
various stages of development. Some are only concepts, others have seen
extensive numerical studies, and other still have been realized experimentally.
However, none of the ideas for novel computing architectures is anywhere close
to becoming a mature technology that could rival CMOS, and there is also no
obvious candidate that could be pushed forward as the single most promising
future technology.

Devices implementing Boolean logic can be characterized by the physical
property---the computational variable---used to represent binary state, as well
as input and output \cite{nikonov2013overview}. For example, the MOSFET uses
charge on the oxide capacitor as its state variable, but voltage for input and
output. Other computational variables include electronic or atomic spin, used in
spintronic and nanomagnetic devices, position, used in some
micro-electro-mechanical approaches, or the electric dipole moment, used for
ferroelectric systems. For Boolean logic devices, binary switches with
characteristics similar to the transistor are usually required, such as gain,
non-reciprocity, i.e.~no feedback from the output to the input, and the ability
to chain the switches. Similarly, benchmarking often concentrates on switching
time and energy, as well as device density. However, if the proposed
architecture is sufficiently dissimilar to CMOS, then these metrics and
requirements become less applicable. As an example, the requirement of
non-reciprocity can be circumvented by introducing clocking schemes; switches
with more than two inputs could potentially perform logic operations
differently, or multiple operations at the same time.

Different materials are being explored to improve the characteristics of the
existing field-effect transistor. For example, III-V compounds such as InAs can
be used for the channel of the transistor to increase electron mobility and
hence switching speeds. Similarly, making the channel a carbon nanotube achieves
nearly ballistic transport, and these devices are then called carbon nanotube
field-effect transistors (CNTFET). As another example, tunnel-junction
field-effect transistors may be used to realize binary switches
\cite{nikonov2013overview} \cite{cavin2012science}. For most of these
approaches, however, accurate and reproducible manufacturing, the integration
with silicon and, not least, the upscaling of fabrication pose severe
challenges. A different route is pursued by replacing the electronic binary
switches with micro-mechanical switches, while still using the same conventional
computing architecture. Micro-electro-mechanical (MEM) relays are relatively
slow, but provide negligible leakage currents and are easy to manufacture,
making them attractive for ultra-low-power applications, such as environmental
sensing logic \cite{kam2011design}. Prototypical MEM circuits have been
experimentally demonstrated \cite{spencer2011demonstration}. Further removed
from CMOS circuitry are spintronic devices, which use the spin degree of freedom
to encode binary information \cite{wolf2001spintronics}. Device proposals cover
a wide range of ideas of how the spin is used, stored, and interfaced with. For
example, domain wall devices represent bits by magnetization domains in
ferromagnetic wires forming a network. The domain walls are propagated through
the wire and junctions, and other geometrical layouts implement logic functions.
The wire's magnetization can then be sensed with a magnetic tunnel junction
\cite{allwood2005magnetic}. Spin wave devices encode information in the phase of
spin waves, which interfere constructively or destructively at junctions, and
multiple signals at different frequencies can potentially be processed in
parallel \cite{khitun2005nano} \cite{kostylev2005spin}. All-spin logic devices
are yet another proposed technology that stores binary state in nanomagnets
which communicate with spin-polarized currents. For logic functionality, a
majority gate has been proposed where spin-polarized currents mix and the
majority spin polarization wins and sets the output \cite{behin2010proposal}
\cite{srinivasan2011all}. 

Quantum-dot cellular automata (QCA) is a beyond-CMOS computing paradigm that is
a more radical departure from conventional CMOS circuit design than most of the
approaches discussed so far \cite{lent1993quantum}. The binary state is encoded
as a bistable charge distribution---electric dipoles---in a cell consisting of
several quantum dots. Cells interact through electrostatic forces in a fashion
similar to a cellular automaton, where each cell's state is dominantly set by
its closest neighbouring cells. The device functionality is determined by the
geometrical arrangement of the cells. For example, cells placed next to each other in
a horizontal line can transport a signal and therefore function as a wire
\cite{lent1993lines}. Multiple input cells placed as closest neighbours to a
fourth cell vote on that cell's state, and the majority wins. This majority gate
is used to realize AND and OR logic; a different geometric arrangement
implements an inverter. \emph{A priori}, the information flow in quantum-dot
cellular automata is not directional. Rather, the computation process can be
understood as perturbing the system out of its ground state by setting external
inputs, where the device then dissipatively propagates to its new ground state,
which corresponds to the computational solution of the problem the circuit was
designed to solve. The approach is current-free and promises extremely low-power
operation. On a higher level, to design large-scale QCA circuits, directionality
in information flow is enforced by introducing a clocking scheme
\cite{lent1997device}. Quantum-dot cellular automata are the subject of this
thesis.

The underlying idea of QCA---bistable interacting cells---is quite versatile and
can be recast in different physical domains. For example, within the last decade
the possibility of molecular QCA implementations has been explored
\cite{lent2000bypassing} \cite{lent2003molecular}. Cells would be comprised of
molecules instead of quantum dots and these molecules have to allow for bistable
electron charge distributions. Due to their molecular scale, these devices
promise to operate at room temperature and allow extremely high device
densities.  Molecular electronics offers the prospect of efficient
self-assembly. However, a molecular QCA scheme also poses some severe
challenges: suitable molecules need to be identified, synthesized reliably,
attached to a surface and arranged in the desired geometric cell layout.
Interfacing input and output with more conventional electronics is likely to be
difficult. A second interesting adaptation of the QCA idea is in the magnetic
domain. Magnetic quantum-dot cellular automata (MQCA) \cite{cowburn2000room}
\cite{bernstein2005magnetic}, occasionally referred to as nanomagnetic logic
(NML) \cite{cavin2012science}, employ bistable nanomagnets as cells which are
coupled through magnetic instead of electrostatic fields. MQCA works at room
temperature, promises very low power dissipation, and is non-volatile. Lines of
cells, the majority gate, and clocking have all been demonstrated experimentally
\cite{imre2006majority} \cite{alam2007clocking} \cite{alam2012chip}.

For the original QCA scheme---sometimes referred to as electrostatic QCA (EQCA)
to distinguish it from the molecular and magnetic variants---a number of systems
have been explored for experimental implementation. Lent \emph{et al}.\
demonstrated the first experimental QCA cell in 1997 in a metal-island system
\cite{orlov1997realization}. The quantum dots were realized as tunnel-coupled
aluminum islands of micrometer size at millikelvin temperatures, and the
bistable nature of the cell was observed.  Experiments were then extended to
demonstrate binary wires (two cells), majority gate operation (a single cell
with three inputs), and a shift register (consisting of six dots)
\cite{orlov1999experimental} \cite{amlani1999digital}
\cite{kummamuru2003operation}. Single QCA cells have also been implemented in
GaAs\,/\,AlGaAs heterostructures, ion-implanted phosphorus-doped silicon, and,
most recently, on a hydrogenated silicon surface \cite{gardelis2003evidence}
\cite{mitic2006demonstration} \cite{haider2009controlled}. On the hydrogenated
silicon surface, individual hydrogen atoms are removed with a scanning
tunnelling microscope tip. The remaining dangling bonds act as quantum dots.
These atomic silicon quantum dots are tunnel-coupled when placed close enough
together (a few nanometers), at larger distances they interact only via Coulomb
repulsion \cite{pitters2011tunnel}. This silicon-based QCA implementation is
particularly exciting, because it promises room temperature operation due to its
small feature sizes and large electrostatic energy scales, and potentially easy
integration with the existing CMOS technology. The precision and upscaling of
the fabrication capabilities have seen encouraging progress recently
\cite{wolkow2013silicon}.

%%%%%%%%%%%%%%%%%%%%%%%%%%%%%%%%%%%%%%%%%%%%%%%%%%%%%%%%%%%%%%%%%%%%%%%%%%%%%%

On the theoretical side, the building blocks of QCA circuitry, such as the
single cell or a line of cells, have been characterized and the dynamical
behaviour of larger systems such as gates has been studied \cite{lent1993lines}
\cite{tougaw1996dynamic}. Importantly, the cell-cell response---the switching
behaviour of a cell with respect to an input cell---was found to be non-linear
and exhibit gain, two of the main requirements for building traditional
CMOS-like integrated logic circuits. If that is indeed true, then lines of cell
are always fully switched and fanout and concatenation of devices do not pose
difficulties.  Clocking schemes have been introduced to improve the reliability
and speed of QCA computations and, starting from the basic building blocks, more
complex circuits like shift registers, adders, and memory have been explored
\cite{lent1997device} \cite{hennessy2001clocking} \cite{rahimi2008quantum}. A
circuit design and simulation program exists, which treats the QCA system with a
high level of abstraction and strongly idealized cells
\cite{walus2004qcadesigner}.

Numerical work on QCA typically starts from an extended Hubbard model. However,
because the full quantum mechanical problem becomes computationally intractable
very quickly even for small systems, two ubiquitous approximations are employed:
the intercellular Hartree approximation (ICHA) and the two-state-per-cell
approximation \cite{lent1993quantum} \cite{tougaw1996dynamic}. Crucially, even
though there are plausibility arguments to motivate their use, neither of these
approximations has been rigorously validated. The ICHA approximation in
particular is problematic: as a mean field scheme ICHA should be expected to
over-emphasize charge-density-wave order in low-dimensional structures and
therefore potentially yield results that are too optimistic regarding the
operational range of the devices. To our knowledge, almost all previous efforts
to characterize QCA building blocks rest on ICHA, and there is a danger that the
whole emerging physical picture of the QCA approach is coloured by the
particularities of this mean field approximation. Recently, Taucer \emph{et
al}.~explicitly identified the need to go beyond the ICHA approximation
\cite{taucer2012consequences}. Concentrating on system dynamics, they showed that
ICHA yields quantitatively and qualitatively wrong results. QCA was found to be
more fragile than previously predicted. Although it has been argued that in
practical systems, quantum decoherence would stabilize QCA, the fact remains
that the approximation underlying most theoretical work on QCA is not well
understood and known to be qualitatively wrong in some cases
\cite{blair2013environmental}.

In this work we undertake a thorough and rigorous numerical study of the
electrostatic QCA approach. We do not attempt the quantitatively accurate
modelling of a specific material system, but aim for the generic, semi-realistic
description of QCA devices. Starting from the extended Hubbard model and using
exact diagonalization, we do away with the ICHA approximation. Instead, we
introduce two controlled Hilbert space truncations, the fixed-charge and the
bond model, whose limits we study and understand. We derive the
two-states-per-cell model---which is equivalent to a transverse-field Ising
model---and show the limits in which it is an appropriate description of QCA
systems. Restricting ourselves to time-independent properties, we concentrate on
a few simple building blocks of QCA---the cell and a line of cells---but aim to
characterize them in as detailed and unbiased a way as possible. Remarkably,
even for these very simple QCA systems, we already find notable differences to
previously published results. In particular, the cell-cell response is
\emph{linear} and does not exhibit gain. This has profound consequences and
essentially changes the whole physical picture of QCA. We explore the systems'
characteristics over a wide range of parameters and establish minimal
requirements for QCA operation as well as parameters for optimal performance.
Using the two-states-per-cell model in a tightly controlled parameter regime, we
briefly investigate wires of up to twelve cells in length and the majority gate.

The following chapter introduces the QCA approach in detail. We explain the
basic idea, logic gates as the building blocks of QCA circuitry, and the
clocking of larger devices. As an example experimental system, we discuss atomic
silicon quantum dots, which we use as a reference throughout the thesis. We then
dive into the modelling of QCA systems, and specifically the extended Hubbard
model which we use as our starting point. Previous theoretical results are
presented, along with a more detailed explanation of the intercellular Hartree
approximation. We conclude the chapter with a brief overview of our exact
diagonalization implementation. Chapter~\ref{ch:approximations} focuses on the
approximations we use. Two Hilbert space truncations are introduced, the
fixed-charge model and the bond model. We then derive an Ising-like model---the
two-states-per-cell approximation---as an effective low-energy model from the
bond Hamiltonian. This derivation will already yield some insights into the
characteristics of the QCA paradigm. The last part of the chapter goes into
great detail to understand how the approximations work and in which regime they
are valid. The fourth chapter presents the numerical results from our study. We
use a three-cell wire as an exemplary QCA system to investigate its basic
characteristics. We then employ an extended ``cluster'' mean field scheme in an
effort to establish lower boundaries for workable QCA system parameters. Using
the Ising model for lines of up to twelve cells, we identify a set of parameters
where the QCA approach works well and put those parameters into context by
contrasting them with corresponding parameter estimates for the atomic silicon
quantum dots. The chapter concludes with a brief numerical exploration of the
majority gate. The last chapter summarizes our results and offers a perspective
on future directions for research on the QCA approach.
